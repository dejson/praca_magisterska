\chapter{Implementacja algorytmu}

Dane pochodzące ze skanowania laserowego wysokiej rozdzielczości zajmują bardzo dużo miejsca, przez co
dostęp do nich poprzez Internet wymaga relatywnie długiego czasu (rzędu kilkunastu sekund). Powoduje to,
iż niemożliwym jest zbudowanie systemu który w czasie rzeczywistym przedstawiałby te dane w sposób podobny
do aktualnie dostępnych w Internecie map, jak Google Maps. Aby zaradzić temu problemowi, stworzono dwa algorytmy.
Ich celem było takie przetworzenie danych wejsciowych, aby zmiejszyć ich objętość jednoczesnie nie tracąc zawartych w nich informacji.
W niniejszym rozdziale znajdują się opisy tych algorytmów, opis przekształcania ich do formatu shapefile w celu udostępnienia
poprzez GeoServer.

\section{Algorytm Naiwny}
