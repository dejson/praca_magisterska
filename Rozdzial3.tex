\chapter{Technologie i systemy web-GIS}
W niniejszym rozdziale zostaną przedstawione protokoły wykorzystywane podczas serwowania map, następnie zostaną omówione serwery umożliwiające przesyłanie map (GeoServer oraz MapServer).
W ostatniej części zostaną przedstawione biblioteki klienckie mogące łączyć się z serwerami i prezentujące mapy użytkownikowi.

\section{Protokoły serwowania map - WMS, WFS, WCS}
W niniejszej sekcji zostaną opisane protokoły serwowania map:
\begin{itemize}
\item WMS - Web Map Service - standard udostępniania map w postaci rastrowej za pomocą protokołu HTTP;
\item WFS - Web Feature Service - standard udostępniania danych przestrzennych w języku znacznikowym GML;
\item WCS - Web Coverage Service - standard udostępniania zmian w danych przestrzennych w czasie.
\end{itemize}

Standardy te zostały opracowane przez Open Geospatial Consortium (OCG). Jest to międzynarodowa organizacja non-profit, skupiająca się na tworzeniu wysokiej jakości standardów dotyczących systemów GIS.

\subsection{WMS}
Web Map Service jest standardem opisującym sposób udostępniania map przez serwer w postaci rastrowej za pomocą protokołu HTTP.
Zgodnie ze standardem \cite{OpenGIS_WMS2006}, rastry te mają być generowane dynamicznie na podstawie danych geograficznych.
Mapy najczęściej są zwracane w jednym z popularnych formatów graficznych, takich jak PNG, GIF lub JPG. Rzadziej jako grafika wektorowa w formatach SVG lub WebCGM.

Standard definiuje 3 dozwolone operacje: pierwsza zwraca informacje opisujące serwer, druga zwraca mapę na podstawie zdefiniowanych parametrów geograficznych, trzecia (opcjonalna) operacja
zwraca informację dotyczące cech obiektów, znajdujących się na mapie. Wszystkie te operacje mogą być wywoływane za pomocą przeglądarki internetowej poprzez URL. Parametry, które należy
podać zależą od rodzaju rządania. W przypadku prośby o dostarczenie mapy są to np: wielkość obrazka wynikowego, część Ziemi która ma zostać zobrazowana czy odwzorowanie. Lista parametrów
została przedstawiona w tabeli \ref{tab:}. Ponadto, standard dopuszcza otrzymywanie poszczególnych map z różnych serwerów. Tym samym, WMS umożliwia stworzenie sieci rozproszonych serwerów
mapowych których klienci mogą tworzyć własne mapy.

%%%%%%%%%%%%%%%%%%%%%%%%%%%%%%%%%%%%%%%%%%%%%%%
TABELA
%%%%%%%%%%%%%%%%%%%%%%%%%%%%%%%%%%%%%%%%%%%%%

Przykładowe zapytanie do serwera WMS wygląda następująco: 

%%%%%%%%%%%%%%%%%%%%%%%%%%%%%%%%%%%
LISTING
%%%%%%%%%%%%%%%%%%%%%%%%%%%%%%%%%%

Wynik takiego zapytania przedstawiono na rysunku \ref{fig:pomorze_gdanskie}.

\begin{figure}[h!]
\label{fig:pomorze_gdanskie}
\end{figure}

\subsection{WFS}

\subsection{WCS}

\section{Geoserver}

\section{MapServer}

\section{OpenLayers}

\section{Leaflet.js}
