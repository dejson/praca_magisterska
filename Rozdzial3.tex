\chapter{Technologie i systemy web-GIS}
W niniejszym rozdziale zostaną przedstawione protokoły wykorzystywane podczas serwowania map, następnie zostaną omówione serwery umożliwiające przesyłanie map (GeoServer oraz MapServer).
W ostatniej części zostaną przedstawione biblioteki klienckie mogące łączyć się z serwerami i prezentujące mapy użytkownikowi.

\section{Protokoły serwowania map - WMS, WFS, WCS}
W niniejszej sekcji zostaną opisane protokoły serwowania map:
\begin{itemize}
\item WMS - Web Map Service - standard udostępniania map w postaci rastrowej za pomocą protokołu HTTP;
\item WFS - Web Feature Service - standard udostępniania danych przestrzennych w języku znacznikowym GML;
\item WCS - Web Coverage Service - standard udostępniania zmian w danych przestrzennych w czasie.
\end{itemize}

Standardy te zostały opracowane przez Open Geospatial Consortium (OCG). Jest to międzynarodowa organizacja non-profit, skupiająca się na tworzeniu wysokiej jakości standardów dotyczących systemów GIS.

\subsection{WMS}
Web Map Service jest standardem opisującym sposób udostępniania map przez serwer w postaci rastrowej za pomocą protokołu HTTP.
Zgodnie ze standardem \cite{OpenGIS_WMS2006}, rastry te mają być generowane dynamicznie na podstawie danych geograficznych.
Mapy najczęściej są zwracane w jednym z popularnych formatów graficznych, takich jak PNG, GIF lub JPG. Rzadziej jako grafika wektorowa w formatach SVG lub WebCGM.

Standard definiuje 3 dozwolone operacje: 
\begin{enumerate}
    \item GetCapabilietes - zwraca informację opisujące serwer (m.in. jego zawartość);
    \item GetMap - zwraca mapę na podstawie zdefiniowanych parametrów geograficznych;
    \item GetFeatureInfo (opcjonalna) - zwraca informację dotyczące cech obiektów, znajdujących się na mapie.
\end{enumerate}
Wszystkie te operacje mogą być wywoływane za pomocą przeglądarki internetowej poprzez URL. Parametry, które należy podać zależą od rodzaju rządania.
W przypadku prośby o dostarczenie mapy są to np: wielkość obrazka wynikowego, część Ziemi która ma zostać zobrazowana czy odwzorowanie. Pełna lista parametrów
została przedstawiona w tabeli \ref{tab:parametry_zapytania_GetMap}. Ponadto, standard dopuszcza otrzymywanie poszczególnych map z różnych serwerów.
Tym samym, WMS umożliwia stworzenie sieci rozproszonych serwerów mapowych których klienci mogą tworzyć własne mapy.

\begin{table}[h!]
    \centering
    \caption{Parametry zapytania GetMap}
    \label{tab:parametry_zapytania_GetMap}
    \begin{tabular}{|p{0.35\linewidth}|p{0.15\linewidth}|p{0.5\linewidth}|}
        \hline
        Parametr & Wymagany & Opis \\
        \hline
        VERSION=1.3.0 & Tak & Wersja serwera z którą chcemy się połączyć \\
        \hline
        REQUEST=GetMap & Tak & Nazwa rządania \\
        \hline
        LAYERS=lista\_warstw & Tak & Oddzielona przecinkami lista 1 lub więcej warstw \\
        \hline
        STYLES=lista\_styli & Tak & Oddzielona przecinkami lista styli (jednego na każdą warstwę) \\
        \hline
        CRS=system\_odniesienia & Tak & Referencyjny system odniesienia \\
        \hline
        BBOX=minx,miny,maxx,maxy & Tak & Rogi prostokąta (lewy dolny, prawy górny), który chcemy otrzymać w jednostkach wybranego systemu odniesienia \\
        \hline
        WIDTH=szerokość & Tak & Szerokość zwróconego obrazka, w pikselach \\
        \hline
        HEIGHT=wysokość & Tak & Wysokość zwróconego obrazka, w pikselach \\
        \hline
        FORMAT=format & Tak & Format zwróconego obrazka \\
        \hline
        TRANSPARENT=TRUE|FALSE & Nie & Przezroczystość tła mapy (domyślnie wyłączone) \\
        \hline
        BGCOLOR=kolor & Nie & Kolor tła w formacie heksadecymalnym (domyślnie 0xFFFFFF - biały) \\
        \hline
        EXCEPTIONS=format\_wyjątków & Nie & Format w jakim mają być zwracane wyjątki przez serwer WMS (domyślnie XML) \\
        \hline
        TIME=czas & Nie & Czas dla jakiego chcemy otrzymać mapę \\
        \hline
        Elevation=wysokość & Nie & Wysokość rządanej warstwy (np: dziura ozonowa na różnej wysokości) \\
        \hline
        Inne wymiary & Nie & Dla niektórych danych mogą być dostępne inne niż domyślne wymiary (np: podczerwone i zwykłe zdjęcia satelitarne) \\
        \hline
    \end{tabular}
\end{table}

Przykładowe zapytanie do serwera WMS wygląda następująco: 

%%%%%%%%%%%%%%%%%%%%%%%%%%%%%%%%%%%
LISTING
%%%%%%%%%%%%%%%%%%%%%%%%%%%%%%%%%%

Wynik takiego zapytania przedstawiono na rysunku \ref{fig:pomorze_gdanskie}.

\begin{figure}[h!]
\label{fig:pomorze_gdanskie}
\end{figure}

\subsection{WFS}

\subsection{WCS}

\section{Geoserver}

\section{MapServer}

\section{OpenLayers}

\section{Leaflet.js}
