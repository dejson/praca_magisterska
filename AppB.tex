\chapter{CD-ROM Content and Application Structure}

CD-ROM contains $\mbox{MATLAB}^{\mbox{\textregistered}}$ .m files that were developed while working on this thesis. In the main directory there is .pdf file with electronic version of this work. In the folder "sourcecode" there are following files:

\begin{itemize}
\item \textbf{wyzn2.m, wyznacznik.m} - files that were used to check whether determinant of recombination matrix has been properly calculated.
\item \textbf{accuracy3.m} - file used for evaluating accuracy of compared methods.
\item \textbf{kka\_accuracy\_norm.m} - file used for comparison of norms of residual matrices.
\item \textbf{pipe\_simulator.m, sym\_rur\_di.m} - both files were used to model leak and then check the models if they are able to produce correct estimates of leak size and location. Former one needs latter to work properly.

\end{itemize}

\section{Programming Environment and Program Description}

As stated above, all programs are in the form of $\mbox{MATLAB}^{\mbox{\textregistered}}$ files and are intended to work with this environment. In first position listed above is program for given matrix computes its determinant. Latter two positions consists of more complex programs, where the inversion of matrix is conducted using methods introduced earlier, where the methods are described in chapter 3. The last position on the list is the core part of the thesis that helps to evaluate the validity of the models, thus it is described briefly in following section.

\section{Pipe Simulator Application}

In order to check, whether the models produces proper results, the results are compared with the ones given by the model described in \cite{keerthi_phd}. General scheme of application is presented on figure \ref{fig:appl_scheme}.

\begin{figure}[ht]
   \centering
   \includegraphics[scale=0.75]{img/ldi_prog_scheme3.png}
   \caption{Scheme of pipe simulator used to evaluate validity of obtained results.}
\label{fig:appl_scheme}
\end{figure}

Application consists of two main subprograms: pipe simulator and leak detection and isolation.

\subsection{Pipe Simulator Program}

First subprogram is based on description provided in chapter 4. First step is to initialize the the pipe based on provided in program initial conditions, leak parameters and physical specification. From the initial conditions the state space model is constructed as in (\ref{eq:stan_pref5}). After the initialization there is a phase of simulation, where the values of $\hat{Q}_{in}, \hat{Q}_{ex}, \hat{P}_{in}$ and $\hat{P}_{ex}$ are determined. The data produced by that subprogram is treated as input data for leak detection and isolation system.

\subsection{Leak Detection and Isolation}

Second subprogram uses the data generated by pipe simulator. At first, there is an initialization to obtain state space model of pipe flow process with the same parameters as in pipe simulator. Further, the inversion of matrices is conducted using two methods introduced in chapter 3. After inversion three models are constructed and simulated in order to obtain estimates of inlet and outlet signals. After determination of these signals, the residuals are generated and processed in order to obtain the parameters of the leak.

\section{Application Manual}

To use the program, please open file "pipe\_simulator.m" it using $\mbox{MATLAB}^{\mbox{\textregistered}}$ version 7.0 or later. To specify the parameters of the pipe, its physical parameters can be changed by editing lines 10 to 20, where $Nd$ is number of the segments, $L$ length of the pipe, $c$ velocity of the sound, $Tl$ time of the leak, $Tr$ rise time of the leak, $zl$ location of the leak and $ql$ size of the leak. All values should be given in basic SI unit. Also, there is a possibility to change the number of simulation steps in line 3. After specification of physical parameters, press F5 button to start the simulation. In the main window of the program there will be information after every 1000 steps. At the end of pipe simulator subprogram the leak detection and identification part will be executed. After the simulation, the leak estimates are displayed in the form of figures. Each figure is properly labeled.
