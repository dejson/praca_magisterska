\chapter{Wst\k{e}p i cel pracy}

Jeszcze kilkadziesiąt lat temu jedynym źródłem wiedzy na temat otaczającej przestrzeni były mapy.
Ze względu na ich charakter oraz szybkie tempo rozwoju (np: infrastruktury drogowej) deaktualizowały
się w zawrotnym tempie. Co więcej, nie zawierały one informacji dotyczących ukształtowania powierzchni.
Oczywiście, poprzez system poziomic możliwe jest określnie wysokości danego punktu, jedakże wiąże się to
z pewną niedokładnością oraz wymaga się, aby punkt znajdował się na powierzchni ziemi. Próżno szukać
map analogowych, zawierających informację o wysokości budynków. Tworzyło to wrażenie niedostatku informacji.

Wraz z rozwojem lotnictwa, a następnie satelit kosmicznych, sytuacja zaczęła sie odwracać. Powstawały coraz to
nowe metody zbierania informacji dotyczących powierzchni ziemi. W dzisiejszym świecie, dzięki istnieniu aplikacji
takich jak Google Maps, możemy zobaczyć nie tylko mapę dowolnego miejsca na Ziemi (i to z niemal dowolną szczegółowością),
ale też obejrzeć zdjęcia lotnicze danego terenu a nawet panoramę z poziomu ulicy. Rozwinęła się też nawigacja - dzięki powstaniu
systemów: amerykańskiego GPS, rosyjskiego GLONASS, a w późniejszym czasie - chińskiego Baidu oraz europejskiego Galileo możliwe
jest ustalenie pozycji w czasie rzeczywistym z dokładnością do kilku centymetrów. Ponadto odbiorniki GPS są niezwykle tanie i
powszechne - można je znaleźć w niemal każdym nowoczesnym smartfonie.
Jednocześnie istnieją projekty badawcze, takie jak
\textit{Copernicus}\cite{webiste:copernicus} który zbiera dane SARowskie czy \textit{ISOK}\cite{website:isok} który zbiera
dane LiDAR.

Rozwija się też sama aparatura badawcza. Dzisiejsze smartfony wykonują zdjęcia w rozdzielczości niedostępnej w najdroższych
aparatach instniejących kilkadziesiąt lat temu. Rośnie też rozdzielczość danych zbieranych w pasmach innych niż widzialne.
Także dane LiDAR są wykonywane w coraz większej rozdzielczości, sięgającej nawet kilkudziesięciu punktów na metr kwadratowy.

Widać wyraźnie, iż diametralnie rośnie ilość danych geograficznych jakie są zbierane. Nie tylko wzarsta sama ilość urządzeń
pomiarowych (satelity, samoloty) ale też sama ilość informacji zawarta na jednym ujęciu. Stawia to nowe wyzwania przed badaczami,
którzy te ogromne ilości danych muszą przetworzyć i udostępnić w przyjazny sposób dla użytkownika.

Szczególnie wymagające jest udostępnianie danych LiDAR w wysokiej rozdzielczości ze względu na ich rozmiar. Plik zawierający
w sobie dane dotyczące obszaru wielkości około $0,35km^2$ waży około 180MB. Przesłanie takiego pliku przez sieć Internet wymaga
czasu kliku, czasem nawet kilkunastu sekund. Stąd też pojawia się idea, aby dane te uprościć, a tym samym zmniejszyć czas potrzebny
na ich przesłanie przez sieć

\section{Cele i teza pracy}

Głównym celem pracy jest stwierdzenie, czy możliwe jest stworzenie systemu który umożliwi dostęp do danych zebranych za pomocą
skanowania laserowego w czasie rzeczywistym. Aby potwierdzić lub zaprzeczyć tezie zostaną zaimplementowane różne algorytmy, których
zadaniem będzie możliwie bezstratna kompresja danych LiDAR. Kompresja będzie polegać na odnajdowaniu specyficznych grup punków, które
w rzeczywistości stanowią fragment tej samej powierzchni. Po zalezieniu otoczki takiej powierzchni, możliwe będzie odrzucenie punktów
znajdujących się w jej środku i pozostawienie tylko wielokąta o kształcie tejże powierzchni, tym samym zmniejszając ilość danych
koniecznych do przesyłania

\section{Przegląd rozdziałów}

W rozdziale drugim zostały omówione sposoby pozyskiwania danych LiDAR oraz formaty plików w jakich przechowywuje się dane geograficzne.
Następnie omówiono rózne algorytmy stosowane w celu przetwarzania chmury punktów.

W rozdziale trzecim omówiono technologie stosowane podczas przesyłania danych geograficznych przez sieć. Rozpoczynając od protokołów,
poprzez usługi serwerowe dostarczające mapy kończąc na bibliotekach klienckich pozwalających przeglądać te dane

W rozdziale czwartym przedstawiono dwa zaimplementowane algorytmy. Omówiono ich zasadę działania oraz wskazano na różnicę między nimi.
Dodatkowo opisano w jaki sposób przekształca się dane do formatu SHP.

W rozdziale piątym przedstawiono wyniki eksperymentów przeprowadzonych zarówno na spreparowanych danych jak i na pochodzących z prawdziwego
skanowania laserowego. Porównano jakość uzyskanych wyników a także czasy przetwarzania.

W rozdziale szóstym przedstawiono ostateczne wnioski z pracy.
