\chapter{Wstęp i cel pracy}

lorem ipsum ....

\section{Wstę teoretyczny}


\begin{equation}
x = \frac{y}{2z^2}
\label{eq:roznice}
\end{equation}
dfsafas fasdf asdf asdf Metoda róznic skończonych przedstawiona na \ref{eq:roznice} zostałą zaczerpnięta z . widać na rys. \ref{fig:wykresy} 

\begin{figure}[h!]
\centering
\includegraphics[width=1\textwidth]{img/params}
\caption{Wykresy ciśnienia i nateżenia przepływu}
\label{fig:wykresy}
\end{figure}

\subsection{Algorytmy szukania ścieżek}


\begin{table}[h]
\centering
\caption{Opis tabeli}
\label{tab:tabela1}
\begin{tabular}{|cl|r|}
\hline
sdf & sdfg       &   \\ \hline
    & \textbf{4} &   \\ \hline
    &            & 5 \\ \hline
\end{tabular}
\end{table}

safdsadfdsa

\section{Cel pracy}

sadfsadf

\section{Układ pracy}

sadfasdf
