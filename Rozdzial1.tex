\chapter{Wst\k{e}p i cel pracy}

Jeszcze kilkadziesiąt lat temu jedynym źródłem wiedzy na temat otaczającej przestrzeni były mapy.
Ze względu na ich charakter oraz szybkie tempo rozwoju (np: infrastruktury drogowej) deaktualizowały
się w zawrotnym tempie. Co więcej, nie zawierały one informacji dotyczących ukształtowania powierzchni.
Oczywiście, poprzez system poziomic możliwe jest określnie wysokości danego punktu, jedakże wiąże się to
z pewną niedokładnością oraz wymaga się, aby punkt znajdował się na powierzchni ziemi. Próżno szukać
map analogowych, zawierających informację o wysokości budynków. Tworzyło to wrażenie niedostatku informacji.

Wraz z rozwojem lotnictwa, a następnie satelit kosmicznych, sytuacja zaczęła sie odwracać. Powstawały coraz to
nowe metody zbierania informacji dotyczących powierzchni ziemi. W dzisiejszym świecie, dzięki istnieniu aplikacji
takich jak Google Maps, możemy zobaczyć nie tylko mapę dowolnego miejsca na Ziemi (i to z niemal dowolną szczegółowością),
ale też obejrzeć zdjęcia lotnicze danego terenu a nawet panoramę z poziomu ulicy. Jednocześnie istnieją projekty takie jak
\textit{Copernicus}\cite{webiste:copernicus} który zbiera dane SARowskie czy \textit{ISOK}\cite{website:isok} który zbiera
dane LiDAR. Mnogość informacji może być przytłaczająca jednocześnie dająć nowe możliwośći przetważania.

\section{Cele i teza pracy}

\section{Przegląd rozdziałów}
