\chapter{Podsumowanie}

Dane LiDAR wysokiej rozdzielczości stanowią bogate źródło informacji.
Ich zbieranie jest mniej czasochłonne, niż mogłoby się wydawać - w ramach projektu
ISOK zeskanowano powierzchnię niemalże całej Polski. Na podstawie tych danych można
tworzyć modele 3D terenu, przydatne przy planowaniu przestrzennym czy do budowania
modeli powodziowych. Nie można jednak pominąć pewnych wad technologii LiDAR. Przede
wszystkim, dane zdobywane poprzez skanowanie laserowe są danymi zdobywanymi na dany
dzień, co wprowadza konieczność ich uaktualniania. Drugą wadą jest ilość miejsca jaką
zajmują dane. Sam plik testowy wykorzystywane do eksperymentów ważył prawie 180MB.
Dla powierzchni całej Polski waga wszystkich danych wynosi 3,5TB. Stanowi to znaczną
trudność przy przesyłaniu takich danych przez sieć

W ramach pracy wykazano, że możliwa jest taka zamiana surowych danych LiDAR, aby
możliwe było ich przesyłanie bez dużych strat informacji. Dane opisujące badany 
obszar na terenie Politechniki Gdańskiej udało się zmniejszyć 180-krotnie, do
wielkości rzędu 1MB. Ponadto, zamieniono je na popularny format SHP, który jest
obsługiwany przez większość oprogramowań serwerowych dla systemów GIS, takich
jak opisywane w tej pracy GeoServer i MapServer. Pozwala to na łatwe udostępnianie
danych poprzez sieć, także z wykorzystaniem protokołów OGC. Dzięki temu
możliwe jest budowanie własnych aplikacji w oparciu o przekonwertowane dane LiDAR.
Jednocześnie sam algorytm wykonuje się stosunkowo szybko (kilkanaście minut). Pozwala
to na cykliczne zbieranie danych i przetwarzanie ich, co podniesie aktualność prezentowanych
danych.

Oczywiście przedstawione w pracy algorytmy i ich realizację ciężko uznać za gotowy
system. Stanowią one raczej swoisty dowód, że stworzenie systemu do udostępniania
danych LiDAR jest możliwe. Wyniki dostarczane przez testowane algorytmy albo nie
umożliwiają jakiekolwiek analizy (algorytm iteracyjny) albo stanowią pewne uproszczenie
dla danych oryginalnych (algorytm naiwny). Uproszczenia te wynikają zarówno z powodu
efektu wypłaszczania (\autoref{chap:upraszczanie}), jak i z niedokładnośći samej
klasyfikacji. Jednakże dla pewnego poziomu szczegółowości, można zaakcpetować te
niedokładności. W ostatecznym systemie dostarczającym dane LiDAR należałoby wykorzystać
zaproponowane algorytmy (lub podobne) do zbudowania modelu uproszczonego, który
byłby wykorzystany przy małym poziomie szczegółowości (dużym oddaleniu od mapy).
Przy dużym poziomie szczegółowości należałoby przesłać do użytkownika oryginalne
dane LiDAR. Moment w którym nastąpi przejscie z modelu uproszczonego na oryginalne
dane powinien być uzależniony od wielkości obszaru, o który prosi użytkownik. Jeżeli
dla zadanej powierzchni możliwe jest przesłanie surowych danych LiDAR w krótkim czasie,
to właśnie te dane należy przesłać. W przeciwnym wypadku konieczne jest zastosowanie
uproszczonego modelu.

Dalsze pracę nad zagadnieniem udostępniania danych LiDAR powinny skupić się na poprawie
wydajności algorytmu iteracyjnego. Wykazuje on tendencję do poprawiania wyników, lecz
niemożliwe było sprawdzenie tego w pełni przez zbyt długi czas wykonywania. Dodatkowo
można rozważyć wykorzystanie innych metryk niż średnia wysokość punktów w zbiorze
i odchylenie standardowe od wysokości. Być może pozwoliłoby to na ograniczenie liczby
koniecznych iteracji a tym samym - zmniejszenie czasu potrzebnego na przetworzenie danych.
